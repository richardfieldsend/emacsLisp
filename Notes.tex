\documentclass[a4paper]{article}

\begin{document}
\section{Notes on Emacs Initialisation November 2017.}
\label{sec:notes-emacs-init}

As with previous versions of these files, the initialisation files are
stored in a GitHub repository.  However, this time I am trying to
install as much as possible from source in order to keep Emacs more
current (my installation of Linux Mint currently provides Emacs 24,
and I am running Emacs 27!)

In order to support the initialisation files in this folder there is a
companion folder where related repositories are stored. These
repositories are:

\begin{description}
\item[auctex] Provide \LaTeX{} document creation support.
\item[autopair] Bracket matching - need to check if this is actually used.
\item[color-theme] Look and feel support - check to see if used
\item[dash] Required by Magit
\item[ebib] Manage BibTeX bibliographies and citations.
\item[emacs-material-theme] Look and feel (hence possibly not needing
  the color-theme)
\item[magit] Git Porcelain for managing content in a repository.
\item[minimap] Display a miniature view of the buffer in a side window.
\item[org-bullets] Fancy bullets display in Org-Mode
\item[org-mode] All encompassing Emacs plugin.
\item[parsebib] required by ebib
\item[rainbow-delimiters] colour multiple levels of nested brackets.
\item[with-editor] also required by magit.
\item[nyan-mode] Add buffer indicator in the shape of Nyan-Cat
\end{description}

Testing signing of a commit.
\end{document}

%%% Local Variables:
%%% mode: latex
%%% TeX-master: t
%%% End:
